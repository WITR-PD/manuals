\documentclass{witrman}

\title{DJ TRAINING MANUAL}
\date{FALL 2018}
\renewcommand{\TitlepageGraphic}{images/titlepage}

\begin{document}

\maketitle

\maketoc

\setpagebg

\chapter{Becoming a DJ at WITR}

DJs are the face of WITR's whole operation, and are the gatekeepers of our
on-air broadcasts.  WITR's weekly show schedule consists of live DJing from
students and community members, who all bring their own diverse tastes in music
and styles to the table.  DJing allows members to discover new music, improve
communication skils, and represent WITR as an organization.  Running the board
is truly a lot of fun, but keep in mind that it also requires preparation and
passion.  This manual will help you to prepare for the responsibilities that a
DJ has at WITR.

The process to become a DJ at WITR consists of several steps, through which your
trainer will guide you:

% MISSING GRAPHIC HERE; USE TIKZ

Shadowing a DJ is meant to introduce you to some of our DJs and to let you
decide whether or not you want to become a DJ at WITR yourself.  After
shadowing, if you have decided that DJing is something you want to pursue, you
must first pass the All Member Test before you can begin training.  DJing is a
privilege, reserved strictly for members of the station.  Once you are
officially a member, the Program Director will pair you with one of our seasoned
DJ trainers.  Your trainer will work with you for one hour sessions once a week
to develop your skills and get you into the groove of DJing.

When they feel you are ready, your trainer will arrange for you to take our
written DJ test.  Everything that will be on the test can be found in this
manual, although you will likely find that after working with your trainer for
several weeks, you will know most of the information already.

Upon passing this test, you will DJ for several one hour, supervised shows to
demonstrate that DJing has become second nature to you.  You can then schedule a
time and day with your trainer to record a two hour DJ demo.  Your trainer will
supervise the Underground feed from the station office, but you will DJ alone in
the studio for these hours.  The Program Director will then review your demo and
grade your performance based on a rubric derived from this training manual,
which takes into account your timing, style, transitions, and more.

After passing your demo, you will officially be authorized to DJ on air in
Studio X!

% INCLUDE GRAPHIC HERE OF WITR WALL

\chapter{Programming}

We are a proudly versatile station in terms of programming.  While WITR 89.7 is
primarily an indie-based station, we also broadcast a variety of genre-specific
specialty shows in addition to shows that feature local artists, artist
interview pieces, sports commentary, and weekly news.  As a DJ, you will have
many opportunities to discover new music and make your Pulse of Music show
unique to you.  No two DJs are quite the same in their musical style choices,
and our listeners get to experience a diverse range of genres as a result.

WITR is not a completely free-form station when it comes to content.  There are
developed rules and guidelines in place to ensure that our DJs and our audience
can enjoy professional, high-quality broadcasting.  It is vital that you adhere
to the following format and are knowledgeble of WITR's on-air requirements when
you're operating the studio.

\section{The Pulse of Music Format}

The Pulse of Music is a mainstay of WITR's daytime and weekday programming.
Every DJ is trained first and foremost in the Pulse of Music format and by
default, when no DJs are on the air, our automation system plays Pulse of Music
content.  This format is what we have determined to be the best balance between
new and interesting content for our listeners, and great creative freedom for
our DJs.  The end result is a professional, catchy broadcast style that
represents the station's tastes as a whole.

A Pulse of Music show is typically 2 hours long, but can be broken up into two
separate, 1-hour long shows if necessary.  Because this is not a specialty show,
only WITR-owned music can be played for these hours.  DJs can select their onw
music from our vast library downstairs, our bin of new albums, and our racks of
local music.  This by no means limits our DJs options, as our library consists
of thousands of CDs from every genre on the spectrum and the second largest
private vinyl collection in New York State.

\section{Components}

There are several different types of content that a Pulse of Music show may
contain.
\begin{skinnyitemize}
    \bolditem{New Bin}{Our new music from the past 3 months.  Albums are
        regularly cycled in and out of the New Bin.}
    \bolditem{Feature}{Particularly awesome tracks that we really want to expose
        our audience to.  Features are decided at open meetings once a week.
        Any member is welcome to attend and contribute their opinions.}
    \bolditem{Recurrent}{This is where feature tracks are moved after they're
        done being featured.  The entire album a feature was on is considered
        recurrent, not just the featured track.}
    \bolditem{Specialty}{Several racks of albums frequently used for specialty
        shows}
    \bolditem{Double Shot}{A ``double shot'' is when you play two songs
        by the same artsit consecutively.  The tracks \textbf{must} be played
        back-to-back, but do not need to be from the same album.}
\end{skinnyitemize}

% INCLUDE GRAPHIC OF NEW BIN HERE
% INCLUDE GRAPHIC OF SPECIALTY SHOW RACK HERE

\section{Structure}

A Pulse of Music show \textbf{must} contain these elements:

\begin{skinnyitemize}
    \item New Bin: at least 50\% of the music you play
    \item Features: 2 per hour
    \item Recurrent: 1 per hour
\end{skinnyitemize}

Except for double shots, the same artist \textbf{should not} be played more than
once every 2 hours.

A double shot can only be played once per hour.  You can also play up to two
specialty show tracks an hour, provided that you follow up these tracks by
mentioning the specialty show they are from, and the day and time that show
occurs.

\section{Imaging}

As you'll know from listening to radio yourself, a broadcast does not consist
solely of music.  The songs are punctuated by various advertising, promotions,
and sound bites, and WITR is no different.  Any audio clip that isn't a song is
called ``imaging.''

Our imaging at WITR is made of audio files that have been cut and edited by our
production department.  These sound clips are incredibly important for the
station's image, for our business department, and for FCC compliance.  Playing
WITR-specific promos lets our listeners know what station they're tuned to and
conveys our station's personality in between sets.  Underwriting spots can be
sold to local businesses and RIT clubs to generate revenue for the station.
Additionally, the FCC requires us to play certain imaging by law.  Ultimately,
WITR uses several different types of imaging:

\begin{skinnyitemize}
        \bolditem{Legal ID}{Our call letters immediately followed by the
        community we serve.  For us, this is ``W-I-T-R Henrietta.'' You must
        pronounce the individual letters of WITR; ``witter'' is only a nickname.
        A legal ID must be played \textbf{once an hour, at the top of every
        hour, give or take 5 minutes}.  The legal ID may also be spoken during a
        mic break, preferably the first natural mic break near the top of the
        hour.}
\end{skinnyitemize}

\end{document}
