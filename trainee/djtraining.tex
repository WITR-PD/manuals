\documentclass{witrman}

\title{DJ TRAINING MANUAL}
\date{FALL 2018}
\renewcommand{\TitlepageGraphic}{images/titlepage}

\begin{document}

\maketitle

\maketoc{}

\setpagebg{}

\chapter{Becoming a DJ at WITR}

DJs are the face of WITR's whole operation, and are the gatekeepers of our
on-air broadcasts.  WITR's weekly show schedule consists of live DJing from
students and community members, who all bring their own diverse tastes in music
and styles to the table.  DJing allows members to discover new music, improve
communication skils, and represent WITR as an organization.  Running the board
is truly a lot of fun, but keep in mind that it also requires preparation and
passion.  This manual will help you to prepare for the responsibilities that a
DJ has at WITR\@.

The process to become a DJ at WITR consists of several steps, through which your
trainer will guide you:

% MISSING GRAPHIC HERE; USE TIKZ

Shadowing a DJ is meant to introduce you to some of our DJs and to let you
decide whether or not you want to become a DJ at WITR yourself.  After
shadowing, if you have decided that DJing is something you want to pursue, you
must first pass the All Member Test before you can begin training.  DJing is a
privilege, reserved strictly for members of the station.  Once you are
officially a member, the Program Director will pair you with one of our seasoned
DJ trainers.  Your trainer will work with you for one hour sessions once a week
to develop your skills and get you into the groove of DJing.

When they feel you are ready, your trainer will arrange for you to take our
written DJ test.  Everything that will be on the test can be found in this
manual, although you will likely find that after working with your trainer for
several weeks, you will know most of the information already.

Upon passing this test, you will DJ for several one hour, supervised shows to
demonstrate that DJing has become second nature to you.  You can then schedule a
time and day with your trainer to record a two hour DJ demo.  Your trainer will
supervise the Underground feed from the station office, but you will DJ alone in
the studio for these hours.  The Program Director will then review your demo and
grade your performance based on a rubric derived from this training manual,
which takes into account your timing, style, transitions, and more.

After passing your demo, you will officially be authorized to DJ on air in
Studio X\@!

% INCLUDE GRAPHIC HERE OF WITR WALL

\chapter{Programming}

We are a proudly versatile station in terms of programming.  While WITR 89.7 is
primarily an indie-based station, we also broadcast a variety of genre-specific
specialty shows in addition to shows that feature local artists, artist
interview pieces, sports commentary, and weekly news.  As a DJ, you will have
many opportunities to discover new music and make your Pulse of Music show
unique to you.  No two DJs are quite the same in their musical style choices,
and our listeners get to experience a diverse range of genres as a result.

WITR is not a completely free-form station when it comes to content.  There are
developed rules and guidelines in place to ensure that our DJs and our audience
can enjoy professional, high-quality broadcasting.  It is vital that you adhere
to the following format and are knowledgeble of WITR's on-air requirements when
you're operating the studio.

\section{The Pulse of Music Format}

The Pulse of Music is a mainstay of WITR's daytime and weekday programming.
Every DJ is trained first and foremost in the Pulse of Music format and by
default, when no DJs are on the air, our automation system plays Pulse of Music
content.  This format is what we have determined to be the best balance between
new and interesting content for our listeners, and great creative freedom for
our DJs.  The end result is a professional, catchy broadcast style that
represents the station's tastes as a whole.

A Pulse of Music show is typically 2 hours long, but can be broken up into two
separate, 1-hour long shows if necessary.  Because this is not a specialty show,
only WITR-owned music can be played for these hours.  DJs can select their onw
music from our vast library downstairs, our bin of new albums, and our racks of
local music.  This by no means limits our DJs options, as our library consists
of thousands of CDs from every genre on the spectrum and the second largest
private vinyl collection in New York State.

\section{Components}

There are several different types of content that a Pulse of Music show may
contain.
\begin{skinnyitemize}
    \bolditem{New Bin}{Our new music from the past 3 months.  Albums are
        regularly cycled in and out of the New Bin.}
    \bolditem{Feature}{Particularly awesome tracks that we really want to expose
        our audience to.  Features are decided at open meetings once a week.
        Any member is welcome to attend and contribute their opinions.}
    \bolditem{Recurrent}{This is where feature tracks are moved after they're
        done being featured.  The entire album a feature was on is considered
        recurrent, not just the featured track.}
    \bolditem{Specialty}{Several racks of albums frequently used for specialty
        shows}
    \bolditem{Double Shot}{A ``double shot'' is when you play two songs
        by the same artsit consecutively.  The tracks \textbf{must} be played
        back-to-back, but do not need to be from the same album.}
\end{skinnyitemize}

% INCLUDE GRAPHIC OF NEW BIN HERE
% INCLUDE GRAPHIC OF SPECIALTY SHOW RACK HERE

\section{Structure}

A Pulse of Music show \textbf{must} contain these elements:

\begin{skinnyitemize}
    \item New Bin: at least 50\% of the music you play
    \item Features: 2 per hour
    \item Recurrent: 1 per hour
\end{skinnyitemize}

Except for double shots, the same artist \textbf{should not} be played more than
once every 2 hours.

A double shot can only be played once per hour.  You can also play up to two
specialty show tracks an hour, provided that you follow up these tracks by
mentioning the specialty show they are from, and the day and time that show
occurs.

\section{Imaging}

As you'll know from listening to radio yourself, a broadcast does not consist
solely of music.  The songs are punctuated by various advertising, promotions,
and sound bites, and WITR is no different.  Any audio clip that isn't a song is
called ``imaging.''

Our imaging at WITR is made of audio files that have been cut and edited by our
production department.  These sound clips are incredibly important for the
station's image, for our business department, and for FCC compliance.  Playing
WITR-specific promos lets our listeners know what station they're tuned to and
conveys our station's personality in between sets.  Underwriting spots can be
sold to local businesses and RIT clubs to generate revenue for the station.
Additionally, the FCC requires us to play certain imaging by law.  Ultimately,
WITR uses several different types of imaging:

\begin{skinnyitemize}
        \bolditem{Legal ID}{Our call letters immediately followed by the
        community we serve.  For us, this is ``W-I-T-R Henrietta.'' You must
        pronounce the individual letters of WITR\@; ``witter'' is only a
        nickname.  A legal ID must be played \textbf{once an hour, at the top of
        every hour, give or take 5 minutes}.  The legal ID may also be spoken
        during a mic break, preferably the first natural mic break near the top
        of the hour.}
        \bolditem{PSA}{A ``Public Service Announcement'' written by the
        National Ad Council.  These are provided to us by the Ad Council and
        come in three different lengths, 15, 30, and 60 seconds.  A PSA (of any
        length) must be played \textbf{once an hour}.}
        \bolditem{Specialty Show Promo}{A promotional clip highlighting any one
        of WITR's weekly specialty shows.  You must play \textbf{one promo per
        hour}.}
        \bolditem{Underwriting}{This is a way for WITR to acknowledge donations
        from businesses without making any ``calls to action.''  Underwriting is
        scheduled for \textbf{specific time slots on certain days} and must be
        played at these times, which will be noted in the weekly logs.}
        \bolditem{Weekend Events}{A summary of upcoming RIT events, usually
        voiced by our own station members and edited by the News Director.  Like
        underwriting, we don't always have a weekend event imaging recorded.
        When we do, they must be played \textbf{once every 2 hours}.}
        \bolditem{Shotgun}{A 2--5 second sound bite that includes ``WITR 89.7''
        or just ``WITR.''  These promote our station and create natural
        transitions between songs.}
        \bolditem{Sweeper}{A 10--30 second sound bite that gives information
        about WITR\@.  There are two types of sweepers: attitude and position.
        Position sweepers reflect the station and its contributions to the
        public.  Attitude sweepers tend to be more tongue-in-cheek.}
\end{skinnyitemize}

The last two imaging types listed are self-promotional; they boost WITR's image
and convey our station's character. DJs can choose which promotional spots to
play at their own discretion, but listeners should hear ``WITR'' roughly
\textbf{once every 8 minutes}.


\section{Specialty Show Format}

The format of a Specialty Show is vastly different from that of a Pulse of Music
show.  Specialty Show DJs can play from a certain artist any number of times,
aren't required to include New Bin content, and only have to play legally
required imaging (PSAs and Legal IDs).  Specialty Shows are specific to a single
genre or theme, and because of this narrowing of music choices, DJs are allowed
to bring their own music to play on air.  Additionally, Specialty Show DJs often
elaborate a great deal on the music they play, which results longer, more
in-depth mic breaks.  Specialty Shows have their own names, imaging promos, and
featured banners on the WITR website.  A few examples of Specialty Shows include
reggae, blues, disco, world, jazz, electronic, punk, metal, and many more.

% INCLUDE GRAPHIC OF SHOW ALBUM ART HERE


\chapter{On-Air Rules and Regulations}

In addition to the general station policies and FCC regulations, DJs must adhere
to certain protocols while operating in the studio.  It is crucial that you know
these procedures by heart, as failure to comply with the following rules can
result in suspension or expulsion.

\subsection{Before Each Show}
\begin{skinnyenumerate}
    \item Sign in on the daily FCC log.
    \item Take a power reading and record it in the FCC log.
    \item Review the imaging log to ensure you know which sound bites to play in
        a given week.
\end{skinnyenumerate}

Don't forget to be respectful of the DJ in the studio before you.  Wait outside
the studio until they have left, and if they are still playing at the time of
your slot's start, politely remind them that it is your hour to DJ\@.  If they
refuse to yield the studio, contact the Program Director.  It is recommended,
that you use any pre-show time to plan a few sets.

\subsection{After Each Show}
\begin{skinnyenumerate}
    \item Announce the next upcoming DJ or show during your last mic break.
    \item Sign off in the daily FCC log.
    \item Make sure that either the next DJ has taken control of the studio, or
        that you have turned on automation and noted this in the FCC logs.
    \item Restore the board settings, studio lights, and studio setup to how you
        found them.
        % Restore the board settings, studio lights, and studio setup using WITR
        % Standard PRocedure (WSPR) № N
    \item Clean up all CDs, records, and other materials that you may have used
        during your show.  Return them to their proper location in the library.
\end{skinnyenumerate}

\subsection{Missing a Show}

It's understandable if you have to miss a show every now and then, but remember
there is a responsibility that comes with having a weekly show.

If you cannot make your time slot, you must do the following:

\begin{skinnyenumerate}
    \item Notify the Program Director as soon as possible.
    \item Let the DJ before you know that you won't be there to take over after
        thier show.
    \item Find a substitute DJ to run your show in your place.  Most DJs do
        this by either posting on Facebook or putting a sign up in the office.
    \item If you can't find a substitute and your show is at least 1 day away,
        voice track the show in Studio B.
\end{skinnyenumerate}

Having a regular show is a privilege.  Depending on scheduling each semester, it
is not unusual for two DJs to want the same time slot.  If you were awarded this
spot, but often miss your show, you are depriving a fellow DJ of a show that
they would love to run.

If your show time begins to pose a conflict for you, inform the Program Director
and request to move your show.  Missing 3 shows without prior approval of the
Program Director will result in a \textbf{loss of on-air privileges}.


\subsection{Suspensions}

Suspendable offenses include:

\begin{skinnyitemize}
    \item Improper sign-off/sign-on.  Do not leave the station without
        automation running.
    \item Not properly keeping the daily FCC logs.
    \item Broadcasting obscenity, profanity, or indecency (including during a
        mic break or musical content).
    \item Playing music that isn't from the WITR library during a Pulse of Music
        show.
    \item Failing to play at least 50\% New Bin on a Pulse of Music show
    \item Missing 3 shows without prior excuse from the Program Director
    \item Behaving inappropriately in the station (including the studios, the
        office, the garage, and the pit).
    \item Not maintaining volume levels (either too loud or too quiet).
    \item Breaking any other programming rules as laid out by station policies
        and the Program Director
\end{skinnyitemize}

If you notice another DJ in violation of these rules, you are required to notify
the Program Director immediately.

Suspensions are issued at the discretion of the Program Director, Chief
Engineer, and General Manager.  Any one of these E-board members is capable of
suspending your on-air privileges.  The duration of a suspension depends on the
severity of the offense, and is also at the discretion of the E-board.

Try to never be on the receiving end of one of these:

% INCLUDE GRAPHIC OF FUCK THE POLICE SUSPENSION FORM HERE

No E-board member \textbf{wants} to give out suspensions, and will automatically
give a DJ the benefit of the doubt.  Everyone at WITR wants to see the station
function seamlessly and for DJs to continue rocking the studio at their freedom.
E-board members will always wait to assign a suspension until they have
confirmed firsthand that an offense transpired.


\chapter{Studio Equipment}

In both Studio A and Studio X, there is a myriad of equipment that enables us to
broadcast.  It is a DJ's job to know how each of these pieces of equipment works
and to regularly operate the studio as a whole.

% INCLUDE GRAPHIC OF BAILEY TOUCHING THE BOARD HERE

Easily the most important piece of equipment the DJs use is the Studio Console,
also colloquially referred to as ``the board.''  Everything that you will
broadcast during a show goes through the board, which consists of many different
parts you will need to understand.  Don't let this intimidate you, though!
Working the board will become second nature to you throughout the course of your
training.

% INCLUDE GRAPHIC OF BOARD WITH LABELS HERE

\subsection{Channels}

The 20 thin, vertical stripes (panels) on the board control different channels,
each of which has a different input source, indicated by the small screen at the
top of the panel.  Channels have a red ``On'' button which sends the channel's
output over the air.  This also initiates playback of a source, meaning that if
you have a CD or vinyl cued up, turning on the channel will begin playing that
CD or vinyl.  Note that pressing the ``On'' button again will not start playing
the source if the channel is already on.

Each panel also has an orange ``Off'' button, which mutes all output from a
channel.  As long as a channel is turned off, no audio will go out over that
channel, regardless of where the fader is.  However, it is poor practice to
leave a channel on when you are not using it, as this leaves you open to
accidentally sending audio out over the air.  After using the \textbf{mic, CD,
vinyl, or line channels}, be sure to \textbf{turn them off}.


\subsection{Assignment Switches}

Above and below each channel's fader is a group of small, circular buttons,
referred to as Assignment Switches.  Each switch has a different function:

\begin{skinnyitemize}
    \bolditem{PGM 1}{Sends to the transmitter}
    \bolditem{PGM 2}{Sometimes used to preview audio}
    \bolditem{PGM 3}{Used to test equipment}
    \bolditem{PGM 4}{Patches through the VoxPro call recorder as well as the
    Access remote unit}
\end{skinnyitemize}

Ask the Chief Engineer for more information about equipment testing and VoxPro
operation.

\begin{itemize}
    \bolditem{Preview}{Turns down the volume of the monitors and pushes a
        channel's audio directly to the headphones.  Press the ``Preview''
        button to listen to a channel before sending it over the air.  To be
        safe, always make sure this channel's fader is down so you don't
        accidentally broadcast the audio.  Also, don't forget to \textbf{turn
        Preview off before attempting to broadcast} from the channel.}

    \bolditem{Talk Back}{Allows the DJ to talk to the guest studio without also
        talking on air.  This feature \textbf{should not be used while the mic
        is on}, as it won't mute the mic.  While a guest's mic is off, hold down
        the ``Talk Back'' button on their channel to speak directly to them.}

    \bolditem{Headphones Knob}{Controls the volume of your headphones.}

    \bolditem{Monitor Knob}{Controls the volume of the studio's speakers.}
\end{itemize}

\subsection{Microphones}

\begin{itemize}
    \bolditem{Studio A Control Room Mic}{This microphone is operated only by the
        DJ running the board, and is in a separate room from the guest mics.
        When the mic is turned on, a light in the hallway also turns on to
        notify other people that you're live on air.}

    \bolditem{Studio A Guest Mics}{These are microphones set aside specifically
        for guest usage.  The DJ in Studio A is in charge of the guest mic
        levels and can turn them on and off.  The guests do not have volume
        faders on their side, only on/off buttons.}

    \bolditem{Studio X DJ Mic}{The DJ mic in Studio X is in the same room as the
        guest mics, which renders the ``Talk Back'' button largely useless.
        When any mic is turned on, a ring around the mic will glow red.  Note
        that there is no on-air light in Studio X, so make sure you close the
        door before going on-air.}

    \bolditem{Studio X Guest Mics}{Like the DJ mic, each of the Studio X guest
        mics have a ring light to indicate whether they are turned on.  Unlike
        in Studio A, the guests have no control over their mics.}
\end{itemize}


\subsection{CD Players}

In each of the studios, there are 3 CD players stacked up which, in order from
top to bottom, are controlled by the channels ``CD1,'' ``CD2,'' and ``CD3.''
After selecting a track, do not press play on the CD player.  Turning on the
channel will do this for you.  When you want to eject a CD, \textbf{press the
``Stop'' button before the ``Eject'' button}.  The CD players will not eject a
CD that is not stopped.  If a CD player stops working (or eats a CD), submit an
Equipment Support Ticket through the online DJ portal (covered later).

The CD players do have subtle operational differences between Studios A and X.
In Studio A, you can skip to a track by pressing the ``Skip'' button, and the
play time count begins at 0.  In Studio X, you can skip to a track by using the
``Tune'' knob, and the play time count begins at the song's duration and counts
down.

% INCLUDE GRAPHICS OF STUDIO X AND A CD PLAYERS HERE

A piece of advice that has been passed down by DJs through the ages: place a
CD's case on top of the player it's in.  That way, you know which case to put it
in when you're done, and you can more easily back-announce the tracks you played.

\subsection{Turntables}

There are two turntables in each studio which DJs use to spin vinyl records.
They are controlled by the channels labeled ``TT1'' and ``TT2.''

% INCLUDE GRAPHIC OF TURNTABLE HERE

To prepare a track to play on air:
\begin{skinnyenumerate}
    \item Switch the turntable on (knob on the lower-left corner) and set it to
        either 33 or 45 RPM, as appropriate for the record.
    \item Put the turntable's channel on preview and \textbf{carefully} drop the
        needle using the cueing level.  Press ``Start'' to play the vinyl and
        stop the record once you hear sound.
    \item Rotate the platter (the disc where the vinyl rests) counterclockwise
        about 180º.  This gives the record time to speed up, and avoids a pitch
        shift at the start of the record.
    \item Turn on the corresponding ``TT'' channel and make sure it is potted
        up.
    \item Manually press ``Start'' on the turntable to play the track.
\end{skinnyenumerate}

\subsection{Dump Button}

This is the DJ's trusty ``Oh s---!'' button.  All shows, with the exception of
live hockey broadcasts, must run on a 5 second delay.  This is to prevent
FCC-inappropriate material from being broadcast over the air.  If you
accidentally speak or play an FCC-violating sound bite, hitting the dump button
catches the broadcast up to real-time, effectively preventing the last 5 seconds
of audio from playing.  (Unfortunately, this doesn't alter the online stream.)
The broadcast will then gradually slow down to achieve a 5 second delay time
again.  This means that the dump button cannot be pressed repeatedly in a short
time to skip more than 5 seconds of audio.

\subsection{Levels}

The soundboard's faders control the levels of the audio being sent over the air.
The computer monitor displays these levels as they occur during a broadcast.

% INCLUDE GRAPHIC OF STUDIO A MONITOR HERE

The red lines mark the maximum volume our broadcasts should have.  Not only is
there a legal limit to our broadcast levels, but our audio quality begins to
degrade quickly above the red line.

The level meters are also a good indicator 

\makefooter{}

\end{document}
