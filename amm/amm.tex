%! program = xelatex

% Instructions for Use:
% 1. Download and install some LaTeX distribution.
%    a. If you're on a Mac, get MacTeX from https://tug.org/mactex
%    b. If you're on Windows, get MikTeX from https://miktex.org
%    c. If you're on a Linux distro, find the LaTeX package in your standard
%    repos and install it.
% 2. Download and install Bebas Neue, which is (almost) WITR's logo typeface:
%    http://www.fontfabric.com/bebas-neue/ (for free).  The TravisCI build will
%    have Alternate Gothic No 2D, which is the official font.
% 3. If you don't already have Century Gothic, I'm not sure where to get you a
%    copy.  Maybe find a friend with a copy of Microsoft Office and bum it off
%    of their computer (if you open the start menu and type "fonts", you'll
%    probably find the folder).
%    If you can't find a copy of Century Gothic, look in the witrman.cls file
%    for instructions on substituting a different font.
% 3. Ensure that you have a copy of witrman.cls and its images folder in the
%    same folder as this file (or, if you're comfortable with LaTeX, in your
%    texmf folder).  It contains some common code and graphics across
%    all of WITR's manuals.
% 4. Open your computer's terminal emulator (cmd.exe, Terminal.app,
%    gnome-terminal, etc.) and cd into the folder where this document is
%    located.
% 5. Run "xelatex amm.tex", then wait for about 15 seconds.  It's a large
%    document, so it takes some time to compile.
% 6. If something broke, send an email to wel2138@rit.edu.  I wrote this, so I
%    \emph{should} be able to help you figure out how to make it work.

% Style Notes

% To save future members from typographic atrocities, please only use the
% \textbf{} macro for emphasis.  Do not put things in all caps.  Do not
% underline them.  Do not italicize them.  Just use bold.

% Also, if you find yourself reverting to dirty hacks to get LaTeX to do what
% you want with the body text, you're probably doing something you shouldn't be.
% Step back and reconsider, and maybe you'll find a better way.

% If you don't mind, please double-space the document.  That is, use two spaces
% after each period.  Those of us who use Vim with sentence movement commands
% will thank you :)

\documentclass{witrman}

% Which semester is this for?
\date{SPRING 2020}
% Which manual is this?
\title{ALL MEMBER MANUAL}
% What color should the WITR logo be on the title page?
\renewcommand{\LogoColor}{yellow}
% What color should the title text be on the title page?
\renewcommand{\TitleColor}{white}

% E-board Names
\newcommand{\wGM}{---}
\newcommand{\wPD}{Arthur Tisseront}
\newcommand{\wCE}{William Leuschner}
\newcommand{\wEC}{Liz Kuhlman}
\newcommand{\wMAL}{Lauren Paige}
\newcommand{\wBD}{Glen Brown}

\begin{document}

\maketitle  % Build the title page

\maketoc{}  % Build the table of contents page

% Add a background image to every page from here on out.
\setpagebg{}

% LaTeX ignores vspace when there is nothing else on the page yet, so we put an
% invisible box at the top to space against.
\mbox{}
% This macro tells LaTeX to adjust vertical spacing.  It's more precise than
% what most Word documents do, which is using <enter><enter><enter>… to get
% something approximating the right space.  The argument is a measurement.
% Regular metric and imperial/customary units are supported in addition to
% typoraphical units like em, pt, and pc.
\vspace{16pt}
\chapter{Welcome to WITR 89.7}

WITR is a volunteer-based organization run by the students of the Rochester
Institute of Technology.  Operating in a professional manner is what keeps WITR
89.7 on the air and allows us to broadcast the best new music to the Rochester
community and beyond.  This manual will provide you with the information you
need to know about WITR’s operations and policies in order to be involved with
the station.

Training to become a member of WITR is organized by the Member at Large.
General training sessions are held weekly during the beginning of each semester.
Training sessions for specific departments are held by their respective
department heads.

The process for becoming a member of WITR is broken down into several steps.
First, you must pass the General Membership Test.  The General Membership Test
is composed of short answer questions based on the content of this manual.  It
is hosted on paper in our station.  Upon successfully passing the test, you are
granted membership to the station.  The General Membership Test can be taken up
to fifteen times per semester, with no less than 24 hours between each attempt.
It is strongly encouraged that you spend an hour or two studying this manual
before attempting the test.

After passing the General Membership Test, you are eligible to complete
department-specific training evaluations.  Training for departments may begin
prior to becoming a member, but the final examination cannot be attempted until
you have passed the General Membership Test.

% The introduction should always be on its own page.
\pagebreak

\section{WITR Facility Information}
% vspace can take negative arguments to reduce space.  \topsep is a measurement
% macro that, among other things, is used in list environments to separate the
% list from the text before it.  Here, the space created by \topsep and \parskip
% is too big, so we counteract it by adding negative \topsep, leaving just
% \parskip.
\vspace{-\topsep}
\begin{tightitemize}
    \bolditem{Station Call Letters}{WITR}
    \bolditem{Broadcast Frequency}{89.7 Megahertz}
    \bolditem{Licensed Power}{910 Watts (Effective Radiated Power)}
    \bolditem{Transmitter Location}{Tower A on Mark Ellingson Hall (50A)}
    \bolditem{Station Type}{Non-commercial / Educational}
    \bolditem{Community Served}{Henrietta}
\end{tightitemize}

\section{Station Contact Information}
\vspace{-\topsep}
\begin{tightitemize}
    \bolditem{WITR Office Number}{(585) 475-2000}  % chktex 8
    \bolditem{WITR Request Line}{(585) 475-2271}   % chktex 8
    \bolditem{WITR Website}{\href{https://witr.rit.edu}{witr.rit.edu}}
    % This parbox here lines up the second line of the address underneath the
    % first one
    \bolditem{Mailing Address}{%
        \parbox[t][2.2em][t]{5cm}{32 Lomb Memorial Drive \\
                                  Rochester, NY 14623}%
    }
\end{tightitemize}

\section{Current Executive Board Members}
\vspace{-\topsep}
% You probably don't need to touch anything here, unless the email addresses
% change.
\begin{tightitemize}
    \bolditem{General Manager}{\wGM{}
        % The backslash makes sure there's a space between the name and the
        % parenthesis.
        \ (\href{mailto:gm@witr.rit.edu}{gm@witr.rit.edu})}
    \bolditem{Program Director}{\wPD{}
        \ (\href{mailto:pd@witr.rit.edu}{pd@witr.rit.edu})}
    \bolditem{Chief Engineer}{\wCE{}
        \ (\href{mailto:engineer@witr.rit.edu}{engineer@witr.rit.edu})}
    \bolditem{Event Coordinator}{\wEC{}
        \ (\href{mailto:ec@witr.rit.edu}{ec@witr.rit.edu})}
    \bolditem{Member at Large}{\wMAL{}
        \ (\href{mailto:mal@witr.rit.edu}{mal@witr.rit.edu})}
    \bolditem{Business Director}{\wBD{}
        \ (\href{mailto:business@witr.rit.edu}{business@witr.rit.edu})}
\end{tightitemize}

\section{The Executive Board}
\subsection{General Manager}
The General Manager oversees the Executive Board, ensures compliance with FCC
and RIT regulations, and assumes the responsibilities of all empty positions
within the station.

% A Brief Primer on Dashes:
% Dashes are actually more complicated than just the one that's on your
% keyboard. There are actually four:
% * Hyphen-Minus: This is the one you type
% * Minus: This is one that's set slightly higher, specifically for math. LaTeX
%       automatically uses this one when you're in math mode.
% * En-dash: A dash that is as wide as the letter "n", to be used when
%       expressing a range of numbers (The Yankees beat the Blue Jays 22--6).
%       To typeset this in LaTeX, type two hyphen-minus dashes, like the example
%       above
% * Em-dash: A dash that is as wide as the letter "m", to be used when breaking
%       up a sentence or when used the way it is below. LaTeX does this with
%       three hyphen-minus dashes.

% Also, bold is done with \textbf{text you want bolded}. It stands for Text
% BoldFace.
\textbf{Manages} --- Executive Board and all members

\subsection{Program Director}
The Program Director oversees all on-air content, organizes DJ shows, and
ensures quality of programming.

\textbf{Manages} --- Assistant Program Director, Music Director, News Director,
Sports Director, Production Director, Production Team

\subsection{Chief Engineer}
The Chief Engineer oversees the technical operation of the station, the
maintenance of all equipment, and the maintenance of the WITR website. He or she
serves as the station’s Chief Operator.

\textbf{Manages} --- Staff Engineers, Internal Developer, Engineering Department

\subsection{Event Coordinator}
The Event Coordinator oversees all WITR events, including music services,
concerts, and fundraising events.

\textbf{Manages} --- Assistant Event Coordinator

\subsection{Member at Large}
The Member at Large oversees training, All Member Meetings, social gatherings,
and membership. He or she is responsible for the mediation and resolution of
conflict issues between station members.

\textbf{Manages} --- All members and trainees

\subsection{Business Director}
The Business Director oversees finances, budgeting, underwriting, branding, and
promotions.

\textbf{Manages} --- Secretary, Underwriting Agents, Promotions Director,
Graphic Designer

A complete list of names, email address, phone numbers, and office hours for the
executive board and their respective officers can be found at the front desk of
the downstairs office.  An abbreviated contact list can be found on the entrance
wall of Studio X.  Most contact information is also listed on the WITR website.

\chapter{General Station Policies}

The following are general station policies that all members must abide by. A
physical copy of these policies must be kept in the \textbf{public file}, which
is located in the front desk cabinets in the downstairs office.

\section{All Member Meeting Attendance}
As a member, you are required to attend All Member Meetings, as outlined in our
Bylaws and Policies and Procedures.  These meetings are intended to keep members
informed as to station events, department progress, and any changes to
organization policies.  All Member Meetings are held \textbf{three times a
semseter} during the academic school year. Members are alerted of All Member
Meetings by email two weeks in advance. Physical reminders are posted in the
station one week in advance.

All members should inform the Member at Large \textbf{at least 24 hours} before
an All Member Meeting if they expect to be absent.  This, however, does not
excuse absences.  Absences can only be waived by contacting the \textbf{Member
at Large} and scheduling an individual meeting to review minutes from the missed
All Member Meeting.  Minutes may be reviewed with either the Member at Large or
the General Manager.  It is the responsibility of the member to schedule meeting
make-ups.  Minutes must be made-up within one week of the missed meeting.

Incomplete homework is not a valid excuse to miss a meeting. Members are
informed of meeting times well in advance, and, as such, are expected to
structure their weekend to include ample time for the completion of homework.

\textbf{One unexcused absence} will result in a suspension of station privileges
until meeting minutes are made-up.  The suspension is inclusive of both station
access and on-air privileges.  \textbf{Three unexcused absences} will result in
loss of membership and expulsion from the station.

\section{Station Hours}
% ~ is a non-breaking space.  It keeps "8 hours" from being split across two
% lines.
Running a successful radio station would not be possible without the time and
active participation of our 100+ student members. Because of this, each member
is required to contribute \textbf{8~hours} of service to the station each
semester in order to maintain membership. Hours can be achieved by helping out
with events, reviewing CDs, performing department-specific tasks, or
participating in service events sponsored by WITR department heads.

% Hyperlinks are done with \href{url}{displayed text}
In order to receive credit for your services, you must record your station hours
online in the Station Hours Form found online at
\href{http://witr.rit.edu/dj}{witr.rit.edu/dj}. Station Hours earned by
reviewing CDs are the one exception to this, as the Music Director records them
automatically. Those who do not meet the 8-hour requirement are subject to a
suspension of privileges.

\section{Station Facility Access}
% Directly typing quotes in LaTeX doesn't work like you expect. You have to use
% two backticks (the key above tab, also has the tilde on it) to start the quote
% and two single quotes to end it. The reason this happens is because it allows
% LaTeX to typeset the quotes such that the ones that start the quote curl up
% and the ones that end it curl down. It's the "Smart Quotes" option in Word's
% Autocorrect box, but you get to control when it happens.
% \@ is a macro that tells LaTeX to use "intersentence spacing," which is the
% typsetter's term for "put the larger space after the period.  ChkTeX suggests
% that you include a \@ whenever you end a sentence with a capital letter, to
% hint to the LaTeX engine where you want the space.
Upon passing the General Membership Test, you will be granted access to the
station and its contents by the Member at Large. As a trainee, you may operate
on-air equipment only with a DJ present. Operating equipment without being a
trainee requires special permission from the Program Director. Access to Studio
C and the garage are granted to designated positions within WITR\@. Access to
server rooms, known as ``The Pit'' and ``Pit X'', is granted only to the Chief
Engineer, the Staff Engineers, and the General Manager.

\section{Guest Policy}
All non-member guests must be signed in by a chaperone member when they arrive.
Members are held responsible for the behavior of any guests that they bring into
the station. Trainees are considered non-members until they pass the General
Membership Test. Members may only have a maximum of two guests in the station at
one time, the only exceptions being if a DJ has been assigned more than two
trainees per training time slot or if a Productions Team member is hosting a
band’s recording session. All guests must leave the station before or with the
member that signed them in. Sign-in sheets are located by the main entrance
doors of both Studio X and the downstairs office.

\section{Reporting Profanities and Other FCC Violations}
If you hear a WITR DJ or member violate FCC regulations, please contact the
General Manager and Program Director immediately. Include the name of the
violator and the time of the alleged violation in your report. Do not ignore the
violation or approach the DJ yourself.

\section{Alcohol and Drug Policy}
Alcohol, drugs, and smoking are \textbf{not allowed} in the WITR offices or
studios \textbf{at any time}. Members found in violation of this rule are
subject to automatic loss of membership and expulsion from WITR\@.

\section{Sexual Misconduct}
Sexual harassment, sexual misconduct, and any violations of Title IX are
\textbf{not} tolerated at WITR\@. In the event that any individual feels that
unwelcome, or inappropriate comments or actions were made in station or over the
air, the responsible person will be dealt with in the strictest manner. This
includes judicial action in accordance with the RIT sexual harassment policy.
Sexual misconduct concerns should be brought to the Member at Large.

\section{Equipment Failures}
% href is the command that makes text into a clickable hyperlink.  The first
% bracketed argument is the url to link to; the second one is the text to
% display on the page.
Equipment failures should be reported using the Equipment Support Ticket System
found in the tabs of \href{https://witr.rit.edu/dj/}{witr.rit.edu/dj}. The
system reports any equipment problems to the Chief Engineer. If the situation is
an emergency warranting immediate attention, contact the Chief Engineer
directly.

\section{Food, Drink, Coats, and Bags}
Coats and bags are allowed under the coat rack in the downstairs office. Food
and drink are allowed only in the \textbf{offices} or \textbf{garage}. Food and
drink are never allowed in the library, studios, or pits. Food trash should be
disposed of in the trash bins located in the hallways outside of the downstairs
office or Studio X.

\section{Clean Up}
It is your responsibility to clean up after yourself. Any time you leave the
station, tidy up the areas that you've used in the offices or studios. Not only
does this help WITR maintain a professional appearance, but it also makes the
place we call home a more comfortable and enjoyable environment. DJs who do not
return CDs and records to their proper locations in the library are subject to a
one-week suspension from on-air privileges.

\section{CD Removal}
CDs and albums from our libraries may \textbf{never} be brought outside of the
station.  The only CDs that are allowed out of the station are the
to-be-reviewed CDs found in the \textbf{review cabinets}. These CDs must be
signed out.  There is a zero tolerance policy for theft. Should you strongly
dislike a CD that you are reviewing, return the CD to the station and allow
another member to review it.

\section{Securing the Station}
If you are the last person the leave the station you must:
\begin{enumerate}
    \item \textbf{Make sure automation is running.} If this is unavailable, you
        must \textbf{immediately} notify the Chief Engineer.
    \item \textbf{Tidy up} any areas of the station you have used and clean up
        any other obvious messes.
    \item \textbf{Close all doors} to the station. This primarily includes the
        doors to the garage and Studio C. Do not lock the door between the
        upstairs office and Studio X.
    \item \textbf{Turn off all of the light switches.} Emergency lights will
        automatically be kept on.
    \item \textbf{Ensure that all guests and non-members leave with you.}
\end{enumerate}

\section{FCC Logs}
The binder in Studio X contains our daily FCC logs. The logs are separated by
days of the week and are kept in \textbf{24-hour time}. When filling out logs,
make sure to write \textbf{legibly} in \textbf{blue or black} ink only.

In each section of the log, you will find:
\begin{itemize}
    \item \textbf{Underwriting Log} --- This sheet contains all of the
        underwriting for the week and the specific times that the underwriting
        should be aired. If there is underwriting scheduled during your show,
        make sure to play that underwriting at the correct time and mark on the
        sheet that you have done so.
    \item \textbf{Power Reading Log} --- Current power output is tracked
        automatically by our transmitter and no longer needs to be recorded by
        DJs.  However, DJs must still record when automation has been turned on
        or off. If you are the first DJ to be on the air after automation or the
        last DJ to be on before automation, you must write ``Auto Off'' or
        ``Auto On'' in the Auto Status column.
    \item \textbf{Music Logs} --- DJs must record the names of the artists and
        songs played during their show, as well as the time they are played.
        Music logs should be completed with our online Song Logger at
        \href{http://logger.witr.rit.edu}{logger.witr.rit.edu}. In the event
        that the Song Logger is malfunctioning, DJs should keep a written record
        in a similar format to the Song Logger and include that written record
        in the FCC logs.
\end{itemize}

Should you make an error while filling out the logs, cross it out with a single
line, place your initials next to the mistake, and date it. You may then fill in
the correct values.

\chapter{Transmitter Operation}
Knowing how to use the transmitter remote control is an essential part of being
a member of WITR\@. All members must be familiar with the basic operation and
emergency procedures surrounding the transmitter. If something goes wrong with
the transmitter, immediately contact the \textbf{Chief Engineer} and
\textbf{General Manager}.

\section{Location}
The transmitter remote control is located in Studio A. As previously stated, our
actual transmitter is located on top of Mark Ellingson Hall (Building 50A/MEH).

\section{Channel Keys}
There are two white buttons located on the transmitter remote. These buttons
function similarly to a TV channel controls in that they change the channel that
is currently active. The left button, with an outline of an up arrow, increases
the current channel while the right button, with an outline of a down arrow,
decreases the current channel. Just as different TV channels show different
things, these channels also show different things. For station operation, a
general member only needs to worry about two channels: Channel 1 and Channel 2.

\section{Channel 1}
Channel 1 is only to be used when turning the transmitter on and off. It
essentially serves as the on/off switch for the transmitter. It is used to
display and manipulate the transmitter’s power source. If the transmitter is on,
it will show a voltage much greater than 0. If not, the reading will show
something around 0.00 volts. Pressing the red ``down'' button will turn the
transmitter off if you are in Channel 1. Pressing the green ``up'' button will
turn the transmitter on is the transmitter is off.

With the recent installation of our new transmitter, the remote retains
functionality, but may not display the current voltage. This is does not
indicate that we are not broadcasting and should not be a cause for concern.

\section{Channel 2}
Channel 2 is used to display and manipulate the transmitter’s output power. In
other words, it displays how strong our current signal is. Pressing the green
``up'' button will increase power while pressing the red ``down'' button will
decrease power.

In summary, we use \textbf{Channel 1 to turn the transmitter on and off} and use
\textbf{Channel 2 to read and adjust the power}.

\section{Status Lights}
Small red status lights can be found on the left side of the remote. They
indicate problems or other information about the transmitter. If any light from
1--8 is on, you are probably not on air and do not have control of the
transmitter. Special attention should be paid to the following three status
lights:
\begin{itemize}
    \item \textbf{Light 1} --- The interlock light indicates that we’ve lost
        contact with the transmitter. This means that someone is on the roof of
        Ellingson. However, the transmitter may still be on despite of this.
        Check the on-air feed in the office to verify that WITR is still
        broadcasting.
    \item \textbf{Light 2} --- Light 2 indicates that the transmitter is turned
        off.
    \item \textbf{Light 4} --- Light 4 indicates that the transmitter is muted.
        It may light up when the transmitter is off.
\end{itemize}

\section{How to Power Up and ``Sign On''}
The following procedure should be followed in order to properly and legally turn
on the transmitter and begin broadcast:
\begin{enumerate}
    \item On the transmitter remote control, select \textbf{Channel 1} by using
        the white channel keys
    \item Push the \textbf{green ``up'' button} once to turn the transmitter on.
        Light 2 should turn off to indicate that the transmitter is back on.
    \item When the transmitter voltage has stopped moving, play the
        \textbf{``FCC Sign On''} audio file found in the playout software. There
        should also be a CD in both stuidos with emergency imaging. As noted
        before, with the installation of our new transmitter, the voltage may
        not be displayed.
    \item Sign in on the FCC daily logs. Indicate the event in the FCC logs by
        writing \textbf{``Power Up''} in the power column.
\end{enumerate}

\section{How to Power Down and ``Sign Off''}
The following procedure should be followed in order to properly and legally turn
off the transmitter and sign off.
\begin{enumerate}
    \item Indicate the power down event in the FCC logs by writing
        \textbf{``Power Down''} in the power column and signing out.
    \item Use exit delay in order to ensure that the entire sign-off file plays.
        On the board, press the \textbf{``Dump''} button, then press
        \textbf{``Exit Delay''}.
    \item Play the \textbf{``FCC Sign Off''} audio file found on the computer.
        There should also be a CD located in both studios with emergency
        imaging.
    \item On the transmitter remote, select \textbf{Channel 1} by using the
        white channel keys.
    \item As soon as the file finishes playing, push the \textbf{red ``down''
        button} to turn the transmitter off. \textbf{Light 2} should light up
        indicating that the transmitter is turned off.
    \item Listen to the on-air office feed to ensure that we are actually
        off-air. You should hear static, and possibly muddled voices.
\end{enumerate}

\section{Adjusting Transmitter Power Levels}
WITR’s legal transmitter output is \textbf{936 watts, plus 5\% or minus 10\%}.
936 watts is the transmitter power required to radiate an Effective Radiated
Power of 910 watts. This allows for error in other parts of the system, such as
remote control calibration and transmitter calibration. If the power is high or
low, complete the following procedure to correct the error:
\begin{enumerate}
    \item Record the current power level in the power column of the FCC logs.
    \item Ensure that the remote control is on \textbf{Channel 2}
    \item Push the \textbf{red ``down'' button} or the \textbf{green ``up''
        button} once and watch the reading. Repeat until the power is at or
        close to 936 watts.
    \item When the power is as close to 936 watts as possible, indicate the
        event by writing \textbf{``Adjusted Power''} in the power column of the
        FCC logs and including the new power level.
\end{enumerate}

As stated before, with the installation of our new transmitter, the current
power levels may not be displaying on the remote control. If you have any
concerns regarding the state of transmitter power levels, contact the Chief
Engineer.

\section{Emergency Alert System}
The Emergency Alert System, referred to as the EAS, is fully automated and run
from the blue unit in the rack inside of The Pit. Should an alert be issued, the
EAS will automatically cut over a broadcast. No recording is necessary, but
\textbf{it is strictly against WITR policy to acknowledge an alert on air}.

\chapter{FCC Rules and Regulations}
The following regulations are set by the FCC and apply to all radio broadcasts
and FM operators. They must be strictly adhered to at all times. Violations are
considered an infraction of Federal Law.

\section{General Regulations}
The primary concern of any FM operator or DJ is keeping the transmitter
operating within the correct parameters. As a certified FM operator, it is your
responsibility to know the laws, treaties, rules, and regulations that currently
govern the station that you are operating. Do not operate any radio transmitter
unless such operation is authorized by a valid station license.

The following FCC rules regarding radio transmissions must be understood and
adhered to at all times:
\begin{enumerate}
    \item It is illegal to willingly interfere with any radio communication or
        signal.
    \item It is illegal to transmit false or deceptive signals or communication
        over radio.
    \item It is illegal to falsely identify a radio station by transmitting a
        call sign that has not been assigned by a proper authority to the
        station. This means that you are \textbf{not} allowed to announce any
        call letters other than WITR\@.
    \item It is illegal to transmit unidentified radio communications or
        signals.
    \item It is illegal without authorization to receive or assist in receiving
        any interstate or foreign communication by radio and use such
        communication or any information contained in that communication for
        personal benefit or the benefit or another unauthorized individual. This
        does not apply to receiving or using the contents of any radio
        communication that is transmitted for the use of the general public,
        that relates to distressed ships, aircrafts, vehicles, or persons, or
        that is transmitted by an amateur radio operator or citizen band radio
        operator.
    \item Obscene material may never be played on air at any time. FCC laws
        allow for the broadcast of indecent or profane material in moderation
        during safe harbor hours. However, it is WITR policy that indecent or
        profane material is \textbf{never} played on air at any time, even
        during safe harbor hours.

        \subsection{Obscenity}
        Obscenity is defined through a three-pronged test:
        \begin{tightenumerate}
            \item The material must be found to have a tendency to excite
                lustful thoughts.
            \item The material must explicity depict or describe, in a patently
                offensive way, sexual conduct as defined by applicable law.
            \item The material, as a whole, must lack literary, artistic,
                political, or scientific value.
        \end{tightenumerate}

        \subsection{Indecency}
        Indecency has any of the following definitions:
        \begin{tightitemize}
            \item The depiction or description is explicit or graphic.
            \item The material dwells on or repeats at length descriptions or
                depictions of sexual or excretory organs.
            \item The material appears to pander or is used to tittilate or
                shock.
        \end{tightitemize}

        \subsection{Profanity}
        Profane language includes those words that are so highly offensive that
        their mere utterance in the context presented may, in legal terms,
        amount to a nuisance. This is inclusive of curses, slurs, and words that
        have no other meaning than something profane. 

        In addition to the above FCC policies, the seven words always considered
        unacceptable to be aired are:
        % The numbers here are explicitly specified because otherwise LaTeX uses
        % letters.
        \textbf{\begin{tightenumerate}
            \item[1.] Shit
            \item[2.] Piss
            \item[3.] Fuck
            \item[4.] Cunt
            \item[5.] Cock Sucker
            \item[6.] Motherfucker
            \item[7.] Tits
        \end{tightenumerate}}

        It is WITR policy that these words are \textbf{never aired} at
        \textbf{any time}, including safe harbor hours.

    \item It is illegal to make slanderous/derogatory remarks about any
        individual, business, organization, or institution.
    \item It is illegal to make derogatory remarks regarding race, gender,
        sexual orientation, or religion.
    \item It is illegal to broadcast a telephone conversation without permission
        from the caller. In addition, WITR policy dictates that aired
        conversations must be approved by the Program Director before being
        aired. Therefore, DJs must have permission from both the caller and the
        Program Director.
\end{enumerate}

\section{Station Identification}
% If the code breaks here when you compile it, you're not using XeLaTeX. Regular
% LaTeX can't handle the plus/minus symbol right there. Also, regular LaTeX
% can't do the Century Gothic font without other changes.
The FCC requires that the station perform a legal ID every hour, as close to the
top of the hour as possible. A legal ID must be performed \textbf{±5 minutes}
from the top of each hour during a break in programming. Ideally, the legal ID
will occur at the first natural break after the hour.

The FCC defines a legal ID as an announcement of the station's \textbf{call
letters} immediately followed by the \textbf{community served}.

Therefore, a legal ID for WITR must contain the words ``WITR Henrietta'' with
nothing separating them. DJs may announce the ID live or play ID imaging from
the computer. When announcing the legal ID, WITR must be pronounced as
individual letters (``W-I-T-R''). ``Witter'' is only our station’s nickname and
is not considered a legal ID\@.

\section{Safe Harbor Hours}
The FCC defines safe harbor hours as 22:00--06:00. However, \textbf{WITR policy
does not recognize safe harbor hours}. Indecent or profane material may
\textbf{not} be played at any time, including safe harbor hours. It is both FCC
and WITR policy that obscene material is \textbf{not} allowed at any time.

\section{Commercials and Calls to Action}
As a non-commercial station, WITR is not allowed to play commercials. It is
illegal under the regulations of the FCC\@. When stating opinions over the air,
you may not allow your comments to turn into a commercial.

According to the FCC, a \textbf{call to action} is what elevates a statement to
a commercial. A call to action occurs when you tell a listener to complete an
action, as in ``go get'', ``go buy'', ``go see'', etc.

Mentioning prices is never allowed unless it is in regards to a WITR or RIT
event. Mentioning commercial entities is legal as long as you do not encourage
listeners to do with business with that entity. Descriptions may be informative,
but not encouraging.

Violations of this regulation occur most often by DJs during mic breaks when
they are announcing concerts or events not hosted by WITR or RIT\@. It is
important for DJs to use good judgment. For example, if you would like to talk
about an upcoming concert from your favorite band, you may state that the
concert exists and give the date and location of that concert. You may not,
however, give the price of tickets or tell listeners to ``go to'', ``attend'',
or ``check out'' the show.

\section{Underwriting}
Although WITR cannot air advertising, we are allowed to sell underwriting.
Underwriting, as defined by the FCC, may include the following:
\begin{tightitemize}
    \item Logograms or slogans that identify but do not promote
    \item Locations
    \item Value neutral descriptions of a product or line of services
    \item Brand and trade names
    \item Product or service listings
\end{tightitemize}

\section{Leaving Transmitter Controls Unattended}
It is \textbf{extremely} illegal to leave transmitter controls unattended. Doing
so could cause dead air, which is a waste of resources and severely frowned upon
in the eyes of the FCC\@. If a person who is supposed to take over for a DJ after
a show does not show up and automation is unavailable, the current DJ must power
down the transmitter.

FCC regulations require that you make transmitter controls inaccessible to
unauthorized people. That means that members must be conscious of closing all
the station doors in order to make the transmitter secure. To aid this, most of
our station doors are kept locked automatically. However, this does mean that
any guests must come with if I you need to step out of the station quickly to
use the bathroom during your show or something to that effect.

In case of an emergency where you must leave the building, such as a bomb threat
or fire alarm, you are first responsible for leaving the transmitter in the
correct state. To secure the station in an emergency you must:
\begin{tightenumerate}
    \item Turn off the transmitter.
    \item Close all doors to the station.
    \item Ensure that no person remains in the station when you leave.
\end{tightenumerate}

Note that it is not necessary to tidy up or turn off the lights in an emergency.

\section{FCC Violations}
The three basic penalties that may occur if you violate FCC regulations are as
follows:
\begin{tightenumerate}
    \item You may \textbf{lose on-air and/or station privileges} at WITR\@.
        % Dollar signs normally start math mode. This one is escaped so it
        % doesn't
    \item WITR may be \textbf{fined up to \$275,000}.
    \item Worst Case: WITR may \textbf{lose its FM license} to broadcast.
\end{tightenumerate}

\section{Suspension Policy}
The Chief Engineer, Program Director, Member at Large, and General Manager are
the only people who can suspend you from station privileges. You may be
suspended for any of the following violations:
\begin{itemize}
    \item Improper sign-off of the transmitter (for example: Leaving your show
        without turning on automation)
    \item Improper FCC daily log keeping
    \item Obscene, indecent, or profane material played on air.
    \item Missing an All Member Meeting and failing to make that meeting up
        within one week
    \item Inadequate WITR station hours
    \item Failing to adhere to any of the other policies defined in this manual,
        WITR’s Policies and Procedures, or WITR’s Bylaws
\end{itemize}

% Guess what that macro does...
Suspensions are issued at the discretion of the four identified e-board members.
The length of the suspension shall be defined, but is left to the discretion of
the officer.  Members can be suspended for 1\textsuperscript{st} time offenses.

\pagebreak
\chapter{Conclusion}
WITR strives to be the best student-run college radio station in the country.
More importantly, WITR is a family of hard-working and diverse students. By
understanding and knowing these rules, you too will have to opportunity to
contribute in your own way to this truly special organization.

\makefooter{}

\end{document}
% Aaaaannd... we're done.
